\chapter{Root Systems}  
\label{root}

The starting point for all \Atlas calculations is a connected,
complex, reductive group. While these are fairly complicated objects,
one of the beautiful features of the theory, and one which makes the
entire \Atlas project possible, is that a reductive group $G$ is 
determined by a very compact set of data: its {\it root datum}. 

In this section we summarize the parts of the theory of root data
which we need. Primary references are Chapters 4-6 of Bourbaki/Lie Groups
\cite{bourbaki_4-6} and  Springer's {\it Linear Algebraic Groups} \cite{springer_book}. We also recommend
the books by Humphreys
\cite{humphreys_lie_algebras} and Borel {\cite{borel_algebraic_groups}.


We begin with a pair $X,\ch X$ of 
free abelian groups of finite rank, together with 
a perfect pairing $\langle\,,\,\rangle:X\times\ch X\rightarrow \Z$.
Suppose $\Delta\subset X,\ch\Delta\subset \ch X$ are finite sets,
equipped with a bijection $\alpha\rightarrow\ch\alpha$.
For $\alpha\in \Delta$ define the reflections
$s_\alpha\in\Hom(X,X)$ and 
$s_{\ch\alpha}\in\Hom(\ch X,\ch X)$ as follows:
$$
\begin{aligned}
s_\alpha(x)&=x-\langle x,\ch\alpha\rangle\alpha \quad (x\in X)\\
s_{\ch\alpha}(y)&=y-\langle \alpha,y\rangle\ch\alpha \quad (y\in \ch X).
\end{aligned}
$$


Following \cite[\S7.4]{springer_book} we define a  {\it root
  datum} to be a quadruple
\begin{equation}
\label{e:rootdatum}
D=(X,\Delta,\ch X,\ch\Delta)
\end{equation}
where $X,\ch X$ are free abelian groups of finite rank, in duality via
a perfect pairing $\langle\,,\,\rangle$, and
$\Delta,\ch\Delta$ are finite subsets of $X,\ch X$, respectively. 
We assume there is a bijection
$\Delta\ni\alpha\mapsto\ch\alpha\in\ch\Delta$ such that for
all $\alpha\in\Delta$,
\begin{equation}
\label{e:bijection}
\langle\alpha,\ch\alpha\rangle=2,\,
s_\alpha(\Delta)=\Delta,\, s_{\ch\alpha}(\ch\Delta)=\ch\Delta.
\end{equation}

Suppose we are given $X,\ch X$ and finite subsets  $\Delta\subset X$ and
$\ch\Delta\subset\ch X$. 
By 
\cite[Lemma VI.1.1]{bourbaki_4-6}
applied to $\Q\langle\Delta\rangle$
and $\Q\langle\ch\Delta\rangle$ there is at most one bijection
$\alpha\mapsto\ch\alpha$ satisfying \eqref{e:bijection}.
Alternatively suppose we are given only a finite subset $\Delta$ of $X$,
satisfying $X\subset\Q\langle\Delta\rangle$. 
By (loc.~cit.) there is at most one subset $\ch\Delta$, and bijection
$\alpha\mapsto\ch\alpha$, satisfying \eqref{e:bijection}.
The condition  $X\subset \Q\langle\Delta\rangle$ holds if and only if the corresponding group is
semisimple.


Suppose $(X,\Delta,\ch X,\ch\Delta)$ is a root datum, and let
$V=X\otimes\R$.  Then $(V,\Delta)$ is a root system in the sense of
\cite[Chapter VI, \S1]{bourbaki_4-6}. Although this definition makes
no reference to a bilinear form on $V$, it follows that there is a
non-degenerate bilinear form $(\,,\,)$ on $V$, unique up to scalar,
such that each $s_\alpha$ is an orthogonal reflection with respect to
$(\,,\,)$. We could then identify $V^*$ with $V$, and for $\alpha\in
\Delta$, identify $\ch\alpha\in V^*$ with $2\alpha/(\alpha,\alpha)\in
V$. However this is a bad idea: it is important to remember that roots
and coroots live in different (dual) spaces.

Suppose $D_i=(X_i,\Delta_i,\ch X_i,\ch\Delta_i)$ ($i=1,2$) are root
data. We say they are isomorphic if there is an isomorphism
$\phi\in\Hom(X_1,X_2)$  satisfying 
$\phi(\Delta_1)=\Delta_2$ and 
$\phi^t(\ch\Delta_2)=\ch\Delta_1$.
Here $\phi^t\in \Hom(\ch X_2,\ch X_1)$ is defined by
\begin{equation}
\label{e:transpose}
\langle\phi(x_1),\ch x_2\rangle_2= \langle x_1,\phi^t(\ch
x_2)\rangle_1\quad (x_1\in X_1,\ch x_2\in X_2^\vee).
\end{equation}



\subsection{Root Datum of a reductive group}

Now suppose $G$ is a connected, complex reductive group (see Chapter
\ref{complexgroups}).

A {\it Cartan subgroup} of $\G$ is a maximal torus, and a {\it Borel
  subgroup} is a maximal, connected solvable subgroup.
All Cartan subgroups are $\G$-conjugate, as are all Borel subgroups.
Furthermore if $B$ is a Borel subgroup, then $B$ contains a Cartan subgroup $H$, 
and any two Cartan subgroups of $B$ are $B$ (not just $\G$) conjugate.

One of the key features of the \Atlas approach is that 
$$
\boxed{\text{We fix }H\subset B\subset G\text{ once and for all}}.
$$
Define 
$$
X^*(H)=\Hom_{\text{alg}}(H,\C^*).
$$
This is a lattice, the {\it character lattice of} $G$
with respect to $H$. Similarly the {\it cocharacter lattice} is
$$
X_*(H)=\Hom_{\text{alg}}(\C^*,H)
$$
Suppose $\alpha\in X^*(H)$ and $\tau\in X_*(H)$.
Define $\langle\alpha,\tau\rangle\in\Z$ by:
$$
\alpha(\tau(z))=z^{\langle\alpha,\tau\rangle}\quad(\text{for
  all }z\in\C^*)
$$
Then $\langle\,,\,\rangle$ is a perfect pairing $X^*(H)\times
X_*(H)\rightarrow\Z$. 

Let $\h=\Lie(H)$ and $\h^*=\Hom_\C(\h,\C)$. Then we have canonical isomorphisms
\begin{equation}
\label{e:XH}
\h\simeq X_*(H)\otimes_\Z\C,\quad
\h^*\simeq X^*(H)\otimes_\Z\C
\end{equation}
where $\h$ is the Lie algebra of $H$ and $\h^*=\Hom(\h,\C)$.
(If $G$ has rank $n$ then $H\simeq (\C^\times)^n$,  $X_*(H)\simeq\Z^n$
and $\h\simeq\C^n$.) 
Let $\Delta=\Delta(G,H)$ be the set of roots of $H$ in $G$, and
$\ch\Delta=\ch\Delta(G,H)$ the corrresponding co-roots. 
Associated to $(G,H)$ is the root datum

\begin{equation}
\label{e:DGH}
D(G,H)=(X^*(H),\Delta,X_*(H),\ch\Delta).
\end{equation}



If $H'$ is another Cartan subgroup then there is an element $g\in G$
so that $gHg\inv=H'$.
Let $D'=(X^*(H'),\Delta(G,H'),X_*(H'),\ch\Delta(G,H'))$ be the
corresponding root datum.
The inverse transpose action on characters induces an isomorphism
$$
(Ad(g)^{t})\inv:X^*(H)\rightarrow X^*(H')
$$
which gives an isomorphism $D(G,H)\simeq D(G,H')$.


Now suppose in addition to $H$ we have chosen a Borel subgroup $B$
containing $H$. Let 
$\Delta^+$ be the corresponding set of positive roots of $\Delta$, with
simple roots $\Pi$.
Then $\Pi^{\vee}=\{\ch\alpha\,|\, \alpha\in\Pi\}$ is a set of
simple roots of $\ch\Delta$, and 

\begin{equation}
\label{e:DbGBH}
D_b(G,B,H)=(X,\Pi,\ch X,\Pi^{\vee})
\end{equation}
is a  {\it based root datum}.
If $H'\subset B'$ are another Cartan and Borel subgroup then
there is a {\it canonical} isomorphism  $D_b(G,B,H)\simeq D_b(G,B',H')$.

Each root datum is the root datum of a reductive
algebraic group, which is determined uniquely up to isomorphism, and
the same holds for based root data.

Note that a connected reductive complex group $G$ of rank $n$ is determined by
a small finite set of data: two sets (of order the semisimple rank of
$G$)
of integral $n$-vectors, subject only
to condition \eqref{e:bijection}, which may be expressed by 
requiring that the matrix of dot products is a Cartan matrix.





\chapter{Root Systems}  
\label{root}

For many
purposes one may identify a connected reductive algebraic 
group with its group of complex points.
For the discussion
of real forms (Section \ref{realgroups}), and to keep
the exposition as elementary as possible, we choose to work with complex
groups. Experts, and those with an interest in other fields, may wish
to convert to the language of algebraic groups where appropriate.

We now describe the parameters for a connected reductive complex
group.  These are provided by {\it root data} and {\it based root
 data}. Good references are the books by Humphreys
\cite{humphreys_lie_algebras} and Springer \cite{springer_book}.


We begin with a pair $X,\ch X$ of 
free abelian groups of finite rank, together with 
a perfect pairing $\langle\,,\,\rangle:X\times\ch X\rightarrow \Z$.
Suppose $\Delta\subset X,\ch\Delta\subset \ch X$ are finite sets,
equipped with a bijection $\alpha\rightarrow\ch\alpha$.
For $\alpha\in \Delta$ define the reflection
$s_\alpha\in\Hom(X,X)$:
$$
s_\alpha(x)=x-\langle x,\ch\alpha\rangle\alpha \quad (x\in X)
$$
and define $s_{\ch\alpha}\in \Hom(\ch X,\ch X)$ similarly.

A {\it root datum} is a quadruple
\begin{equation}
\label{e:rootdatum}
D=(X,\Delta,\ch X,\ch\Delta)
\end{equation}
where $X,\ch X$ are free abelian groups of finite rank, in duality via
a perfect pairing $\langle\,,\,\rangle$, and
$\Delta,\ch\Delta$ are finite subsets of $X,\ch X$, respectively. 
We assume there is a bijection
$\Delta\ni\alpha\mapsto\ch\alpha\in\ch\Delta$ such that for
all $\alpha\in\Delta$,
\begin{equation}
\label{e:bijection}
\langle\alpha,\ch\alpha\rangle=2,\,
s_\alpha(\Delta)=\Delta,\, s_{\ch\alpha}(\ch\Delta)=\ch\Delta.
\end{equation}

Suppose we are given $X,\ch X$ and finite subsets  $\Delta\subset X$ and
$\ch\Delta\subset\ch X$. 
By 
\cite[Lemma VI.1.1]{bourbaki_4-6}
applied to $\Q\langle\Delta\rangle$
and $\Q\langle\ch\Delta\rangle$ there is at most one bijection
$\alpha\mapsto\ch\alpha$ satisfying \eqref{e:bijection}.
Alternatively suppose we are given only a finite subset $\Delta$ of $X$,
satisfying $X\subset\Q\langle\Delta\rangle$. 
By (loc.~cit.) there is at most one subset $\ch\Delta$, and bijection
$\alpha\mapsto\ch\alpha$, satisfying \eqref{e:bijection}.
The condition  $X\subset \Q\langle\Delta\rangle$ holds if and only if the corresponding group is
semisimple.


Suppose $D_i=(X_i,\Delta_i,\ch X_i,\ch\Delta_i)$ ($i=1,2$) are root
data. We say they are isomorphic if there is an isomorphism
$\phi\in\Hom(X_1,X_2)$  satisfying 
$\phi(\Delta_1)=\Delta_2$ and 
$\phi^t(\ch\Delta_2)=\ch\Delta_1$.
Here $\phi^t\in \Hom(\ch X_2,\ch X_1)$ is defined by
\begin{equation}
\label{e:transpose}
\langle\phi(x_1),\ch x_2\rangle_2= \langle x_1,\phi^t(\ch
x_2)\rangle_1\quad (x_1\in X_1,\ch x_2\in X_2^\vee).
\end{equation}

Let $G$ be a connected reductive complex group and choose a
Cartan subgroup $H$ of $G$.
Let $X^*(H),X_*(H)$ be the character and co-character lattices of $H$
respectively.
We have canonical isomorphisms
\begin{equation}
\label{e:XH}
\h\simeq X_*(H)\otimes_\Z\C,\quad
\h^*\simeq X^*(H)\otimes_\Z\C
\end{equation}
where $\h$ is the Lie algebra of $H$ and $\h^*=\Hom(\h,\C)$.
(If $G$ has rank $n$ then $H\simeq (\C^\times)^n$,  $X_*(H)\simeq\Z^n$
and $\h\simeq\C^n$.) 
Let $\Delta=\Delta(G,H)$ be the set of roots of $H$ in $G$, and
$\ch\Delta=\ch\Delta(G,H)$ the corrresponding co-roots. 
Associated to $(G,H)$ is the root datum

\begin{equation}
\label{e:DGH}
D(G,H)=(X^*(H),\Delta,X_*(H),\ch\Delta).
\end{equation}



If $H'$ is another Cartan subgroup then there is an element $g\in G$
so that $gHg\inv=H'$.
Let $D'=(X^*(H'),\Delta(G,H'),X_*(H'),\ch\Delta(G,H'))$ be the
corresponding root datum.
The inverse transpose action on characters induces an isomorphism
$$
(Ad(g)^{t})\inv:X^*(H)\rightarrow X^*(H')
$$
which gives an isomorphism $D(G,H)\simeq D(G,H')$.

Now suppose in addition to $H$ we have chosen a Borel subgroup $B$
containing $H$. Let 
$\Delta^+$ be the corresponding set of positive roots of $\Delta$, with
simple roots $\Pi$.
Then $\Pi^{\vee}=\{\ch\alpha\,|\, \alpha\in\Pi\}$ is a set of
simple roots of $\ch\Delta$, and 

\begin{equation}
\label{e:DbGBH}
D_b(G,B,H)=(X,\Pi,\ch X,\Pi^{\vee})
\end{equation}
is a  {\it based root datum}.
If $H'\subset B'$ are another Cartan and Borel subgroup then
there is a {\it canonical} isomorphism  $D_b(G,B,H)\simeq D_b(G,B',H')$.

Each root datum is the root datum of a reductive
algebraic group, which is determined uniquely up to isomorphism, and
the same holds for based root data.

Note that a connected reductive complex group $G$ of rank $n$ is determined by
a small finite set of data: two sets (of order the semisimple rank of
$G$)
of integral $n$-vectors, subject only
to condition \eqref{e:bijection}, which may be expressed by 
requiring that the matrix of dot products is a Cartan matrix.



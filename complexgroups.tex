\chapter{Complex Reductive Groups}  
\label{complexgroups}

The groups at the core of the atlas project are connected, complex
reductive algebraic groups. The Lie groups we are interested in are
the {\it real points} of such a group, which we discuss in Chapter
\ref{realgroups}.

Before giving the formal defintion we give a number of examples.

\begin{example}[General and special linear groups]
Let $\GL(n,\C)$ be the group of invertible, $n\times n$ matrices over $\C$.
Of particular importance is $GL(1,\C)\simeq \C^*$.

The center of $\GL(n,\C)$ is isomorphis to $GL(1,\C)$, and its derived group is $\SL(n,\C)=\{g\in \GL(n,\C)\mid \det(g)=1\}$.

\end{example}

Throughout this section all of our groups are complex, and we drop the
$\C$ from the notation, for example $\GL(n)=\GL(n,\C)$. 

\begin{example}[Special orthogonal and symplectic groups]

The special orthogonal group is
$$
\SO(n)=\{g\in \SL(n)\mid gg^t=I\}
$$

Let $J=
\begin{pmatrix}
  &I_n\\-I_n&0
\end{pmatrix}$. Then the symplectic group is:

$$
\Sp(2n)=\{g\in \SL(n)\mid gJg^t=J\}
$$
\end{example}

The {\it classical groups} are $\GL(n), \SL(n), \SO(n)$ and $\Sp(2n)$.

We obtain other groups from these by taking quotients by a central
subgroup, or passing to the simply connected cover.

A reductive group is said to be {\it of adjoint type} (or simply {\it
  adjoint}) if its center is trivial. If $G$ is reductive, then
$G/Z(G)$ is adjoint, and is denoted $\Gad$.

Here are the classical groups, their centers, and their adjoint groups.

\bigskip

\begin{tabular}{|c|c|c|c|}
\hline
$G$& $n$ & $Z(G)$ & $\Gad$\\\hline
$\GL(n)$ & $n\ge 1$ &$\GL(1)$ & $\PGL(n)$ \\\hline
$\SL(n)$ & $n\ge 2$& $\Z/n\Z$ & $\PSL(n)\simeq \PGL(n)$ \\\hline
$\SO(n)$ & $n\text{ odd}\ge 3$ &$1$ & $\PSO(n)\simeq \SO(n)$\\\hline
 & $n\text{ odd}\ge 3$ &$1$ & $\PSO(n)\simeq \SO(n)$\\\hline
& $n\text{ even}\ge 4$& $\Z/2\Z$ & $\PSO(n)$\\\hline
$\Sp(2n)$ & $n\ge 1$ & $\Z/2\Z$ & $\PSp(2n)$\\\hline
\end{tabular}
\bigskip

We could also have included $SO(2)\simeq GL(1)$.
There are a small number of overlaps in the table, for example $SL(2)\simeq Sp(4)$. 
Typically the adjoint
group does not have a simpler realization than as a quotient, except in low dimensions. 
For example 
$\PSL(2)\simeq \PGL(2)\simeq \SO(3)$, and $\PSp(4)\simeq \SO(5)$. 

The groups $\SL(n)$ and $\Sp(2n)$ are simply connected. The fundamental
group of $\SO(n)$ ($n\ge 3$) is $\Z/2\Z$, so it has a two-fold, simply connected cover, denoted $\Spin(n)$. 

By {\it complex algebraic group} we mean a subgroup $G$ of $GL(n,\C)$
which is the set of solutions of a (finite) set of equations, which
are polynomials in the coordinates. Such a group is said to be {\it
  unipotent} if it is conjugate in $GL(n,\C)$ to a subgroup of
the upper triangular matrices with ones on the diagonal.

The {\it unipotent radical} $\Gu$ of $G$ is the (unique) maximal,
connected, normal, unipotent subgroup of $G$.  The {\it radical}
$\Grad$ of $G$ is the (unique) maximal, connected, normal solvable (as
an abstract group) subgroup of $G$. 

\begin{definition}
A complex, connected reductive group $\G$ is a connected, complex algebraic group such that $\Gu$ is trivial. 
We say $\G$ is {\it semisimple} if  $\Grad=1$.
\end{definition}

By a torus we mean a group isomorphic to $\GL(1,\C)^n$. We denote the derived group of $G$ by $\Gder$. 

Here are some basic properties of connected reductive groups. 

A connected complex group is reductive if and only if it is the complexification of a compact connected group [reference?].


Suppose $\G$ is a complex, connected reductive group. 
\begin{enumerate}
\item $\Lie(G)$ is a reductive Lie algebra.

\item The radical  $\Grad$ of $\G$ is the maximal, connected central torus. 

\item The groups  $\G/\Grad$, $\Gad=\G/Z(\G)$  and $\Gder$ are semisimple.

\item If $A$ is any finite subgroup of $Z(G)$ then $G/A$ is a connected, reductive group.
In particular this holds for $\Gad$.

\item $\G$ is semisimple if and only if $\Lie(G)$ is a semisimple Lie algebra.

\item Suppose $\G$ is semisimple. Then the 
simply  connected cover $\Gsc$ of $\G$, in the sense of Lie groups, is a connected, reductive algebraic group.
In particular $\G\simeq \Gsc/A$ where $A$ is a finite subgroup of $Z(\Gsc)$. 

\end{enumerate}

\begin{lemma}
\item If $G$ is a connected, complex reductive algebraic group then
$G=\Grad\Gder$: every element of $G$ can be written $g=zh$ with
$z\in \Grad$ a $h\in\Gder$, and $(zh)(z'h')=(zz')(hh')$.
\end{lemma}



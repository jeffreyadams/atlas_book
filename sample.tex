%edited starting 8/27/17 for re-submission to Duke
\documentclass[10pt,leqno]{article}
\usepackage{verbatim}
\usepackage{amssymb}
\usepackage{mathtools}
\usepackage{amsrefs}
\usepackage{rotating}
\usepackage{amsmath}
%\usepackage{showkeys}
\usepackage{tabularx}
\setlength\extrarowheight{4pt}   %spacing in tables
\usepackage{theorem}
\usepackage[matrix,tips,frame,color,line,poly,curve]{xy}
\renewcommand{\labelenumi}{(\arabic{enumi})}
\newcommand\kappaarrow[2]{#1\overset\kappa\rightarrow#2}
\newtheorem{theorem}[equation]{Theorem}
\newtheorem{corollary}[equation]{Corollary}
\newtheorem{definition}[equation]{Definition}
\newtheorem{lemma}[equation]{Lemma}
\newtheorem{desideratum}[equation]{Desideratum}
\newtheorem{conjecture}[equation]{Conjecture}
\newtheorem{proposition}[equation]{Proposition}
\newtheorem{remark}[equation]{Remark}
{\theorembodyfont{\rmfamily}\newtheorem{theoremplain}[equation]{Theorem}
\newtheorem{remarkplain}[equation]{Remark}
\newtheorem{editorialremarkplain}[equation]{Editorial Remark}
\newtheorem{exampleplain}[equation]{Example}
\newtheorem{corollaryplain}[equation]{Corollary}
\newtheorem{mytable}[equation]{Table}
}

\renewcommand{\sec}[1]{\section{#1}
\renewcommand{\theequation}{\thesection.\arabic{equation}}
  \setcounter{equation}{0}}
\newcommand{\subsec}[1]{\subsection{#1}
\renewcommand{\theequation}{\thesubsection.\arabic{equation}}
  \setcounter{equation}{0}}

\newcommand{\subsubsec}[1]{\subsubsection{#1}
\renewcommand{\theequation}{\thesubsection.\arabic{equation}}
  \setcounter{equation}{0}}

% Danger, Will Robinson!
\def\danger{\begin{trivlist}\item[]\noindent%
\begingroup\hangindent=3pc\hangafter=-2%\clubpenalty=10000%
\def\par{\endgraf\endgroup}%
\hbox to0pt{\hskip-\hangindent\dbend\hfill}\ignorespaces}
\def\enddanger{\par\end{trivlist}}

\newcommand{\Gext}{\negthinspace\negthinspace\phantom{a}^\delta G}
\newcommand{\thetaG}{\negthinspace\negthinspace\phantom{a}^\theta
  G(\C)}
\newcommand{\thetaK}{\negthinspace\negthinspace\phantom{a}^\theta K(\C)}
\newcommand{\qed}{\hfill $\square$ \medskip}
\newenvironment{proof}[1][Proof]{\noindent\textbf{#1.} }{\qed}
\newcommand\exact[3]{1\rightarrow #1\rightarrow #2\rightarrow #3\rightarrow1}
\newcommand{\Aut}{\mathrm{Aut}}
\newcommand{\Inv}{\mathrm{Invol}}
\newcommand{\sgn}{\mathrm{sgn}}
\newcommand{\diag}{\mathrm{diag}}
\newcommand{\gr}{\mathrm{gr}}
\newcommand{\Out}{\mathrm{Out}}
\newcommand{\Int}{\mathrm{Int}}
\renewcommand{\int}{\mathrm{int}}
\newcommand{\Hom}{\mathrm{Hom}}
\newcommand{\Ad}{\mathrm{Ad}}
\newcommand{\zinv}{\mathrm{inv}}
\newcommand{\SRF}{\mathrm{SRF}}
\newcommand{\Gad}{G_\mathrm{ad}}
\newcommand{\Gsc}{G_\mathrm{sc}}
\newcommand{\Zsc}{Z_\mathrm{sc}}
\newcommand{\Ztor}{Z_\mathrm{tor}}
\newcommand{\Gbar}{\overline G}
\newcommand{\Kad}{K_\mathrm{ad}}
\newcommand{\Gal}{\mathrm{Gal}}
\newcommand{\Norm}{\mathrm{Norm}}
\newcommand{\Cent}{\mathrm{Cent}}
\newcommand{\I}{\mathcal I}
\renewcommand{\O}{\mathcal O}
\newcommand{\R}{\mathbb R}
\newcommand{\C}{\mathbb C}
\newcommand{\Z}{\mathbb Z}
\newcommand{\W}{\mathbb W}
\newcommand{\Ztwo}{\mathbb Z_2}
\newcommand{\N}{\mathcal N}
\newcommand{\Q}{\mathbb Q}
\newcommand{\E}{\mathbb E}
\newcommand{\G}{G}
\renewcommand{\H}{\mathbb H}
\newcommand{\h}{\mathfrak h}
\renewcommand{\sl}{\mathfrak s\mathfrak l}
\renewcommand{\P}{\mathfrak p}
\renewcommand{\a}{\mathfrak a}
\newcommand{\zk}{\mathfrak z_\mathfrak k}
\newcommand{\A}{\mathbb A}
\newcommand{\K}{\mathcal K}
\newcommand{\B}{\mathcal B}
\renewcommand{\k}{\mathfrak k}
\newcommand{\spint}{\widetilde{Spin}}
\newcommand{\ch}[1]{#1^\vee}
\newcommand\sigmaqc{\sigma_{\text{qc}}}
\newcommand\thetaqc{\theta_{\text{qc}}}
\newcommand{\Fgal}{F_{\text{gal}}}
\newcommand{\Falg}{F_{\text{alg}}}
\newcommand{\cl}{\mathit{cl}}
\newcommand{\Lie}{\mathrm{Lie}}
\newcommand{\opp}{\text{-opp}}

\renewcommand{\t}{\mathfrak t}
\newcommand{\su}{\mathfrak s\mathfrak u}
\newcommand{\g}{\mathfrak g}
\newcommand\inv{^{-1}}
\newcommand\wh{\widehat}
\newcommand{\GL}{\text{GL}}
\newcommand{\SL}{\text{SL}}
\newcommand{\SO}{\text{SO}}
\newcommand{\SU}{\text{SU}}
\newcommand{\Spin}{\text{Spin}}

\usepackage{courier}
\begin{document}
\title{Galois and Cartan Cohomology of Real Groups}
\author{Jeffrey Adams and Olivier Ta\"ibi}
\maketitle

{\renewcommand{\thefootnote}{} 
\footnote{2000 Mathematics Subject Classification: 11E72 (Primary), 20G10, 20G20}
\footnote{Jeffrey Adams is supported in part by  National Science
Foundation Grant \#DMS-1317523}
\footnote{Olivier Ta\"ibi is supported by ERC Starting Grant 306326.}}


\section*{Abstract}
Suppose $G$ is a complex, reductive algebraic group. A real form of
$G$ is an antiholomorphic involutive automorphism $\sigma$, so
$G(\R)=G(\C)^\sigma$ is a real Lie group.  Write $H^1(\sigma,G)$ for
the Galois cohomology (pointed) set $H^1(\text{Gal}(\C/\R),G)$.  A
Cartan involution for $\sigma$ is an involutive holomorphic
automorphism $\theta$ of $G$, commuting with $\sigma$, so that
$\theta\sigma$ is a compact real form of $G$.  Let $H^1(\theta,G)$ be
the set $H^1(\Ztwo,G)$ where the action of the nontrivial element of
$\Ztwo$ is by $\theta$.  By analogy with the Galois group we refer to
$H^1(\theta,G)$ as Cartan cohomology of $G$ with respect to $\theta$.
Cartan's classification of real forms of a connected group, in terms
of their maximal compact subgroups, amounts to an isomorphism
$H^1(\sigma,\Gad)\simeq H^1(\theta,\Gad)$ where $\Gad$ is the adjoint
group.  Our main result is a generalization of this: there is a
canonical isomorphism $H^1(\sigma,G)\simeq H^1(\theta,G)$.

We apply this result to give simple proofs of some well
known structural results: the Kostant-Sekiguchi correspondence of
nilpotent orbits; Matsuki duality of orbits on the flag
variety; conjugacy classes of Cartan subgroups; and structure of the
Weyl group.  We also use it to compute $H^1(\sigma,G)$ for all simple,
simply connected groups, and  to give a cohomological
interpretation of strong real forms. For  the applications it 
is important that we do not assume $G$ is connected.

\sec{Introduction}

Suppose $G$ is a complex, reductive algebraic group.  A {\em real form} of
$G$ is an antiholomorphic involutive automorphism $\sigma$ of $G$, so
$G(\R)=G(\C)^\sigma$ is a real Lie group. See Section \ref{s:real} for more details.
Let $\Gamma=\Gal(\C/\R)$
and write $H^i(\Gamma,G)$ for the Galois cohomology of $G$ (if $G$ is
nonabelian $i\le 1$).  If we want to specify how the nontrivial
element of $\Gamma$ acts we will write $H^i(\sigma,G)$.  The
equivalence (i.e. conjugacy) classes of real forms of $G$, which are
inner to $\sigma$ (see Section \ref{s:real}), are parametrized by
$H^1(\sigma,\Gad)$ where $\Gad$ is the adjoint group.

On the other hand, at least for $G$ connected, Cartan classified the
real forms of $G$ in terms of holomorphic involutions as follows. We
say a {\em Cartan involution} for $\sigma$ is a holomorphic involutive
automorphism $\theta$, commuting with $\sigma$, so that
$\sigma^c=\theta\sigma$ is a compact real form.
If $G$ is connected then $\theta$
exists, and is unique up to conjugacy by $G^\sigma$. 
Following Mostow we  prove a similar result in general. See Section \ref{s:real}.

Let $H^i(\Ztwo,G)$ be the group cohomology of $G$ where the nontrivial
element of $\Ztwo=\Z/2\Z$ acts by $\theta$.  As above we denote this
$H^i(\theta,G)$, and we refer to this as Cartan cohomology of $G$.
Conjugacy classes of involutions which are inner to $\theta$ are
parametrized by $H^1(\theta,\Gad)$.

Thus the equivalence of the two classifications of real forms amounts
to an isomorphism (for connected $G$) of the first Galois and Cartan cohomology spaces
$H^1(\sigma,\Gad)\simeq H^1(\theta,\Gad)$.  It is natural
to ask if the same isomorphism holds with $G$ in place of $\Gad$. 
For our applications it is helpful to know the result for disconnected groups as well. 

\begin{theorem}
\label{t:main}
Suppose $G$ is a complex, reductive algebraic group (not necessarily
connected), and $\sigma$ is a real form of $G$.
Let $\theta$ be a  Cartan involution for $\sigma$. 
Then there is a canonical isomorphism $H^1(\sigma,G)\simeq H^1(\theta,G)$. 
\end{theorem}

The proof will be given in Section \ref{s:borelserre}.

The interplay between the $\sigma$ and $\theta$ pictures plays a
fundamental role in the structure and representation theory of real
groups, going back at least to Harish Chandra's formulation of the
representation theory of $G(\R)$ in terms of $(\g,K)$-modules.  The
theorem is an aspect of this, and we give several applications.

Suppose $X$ is a homogeneous space for $G$, equipped with a real structure
\enddocument
\end
%%% Local Variables: 
%%% mode: latex
%%% TeX-master: t
%%% End: s_{1}:0 -> -1 6



\chapter{Automorphisms of Reductive Groups}
\label{automorphisms}

Throughout this section $\G$ is a connected reductive complex group.
We write  $\Aut(\G)$ for the group of algebraic (equivalently holomorphic) automorphisms of $\G$. 

If $g\in\G$ then $\int(g)$ denotes the inner automorphism of
conjugation by $g$: $\int(g)(h)=ghg\inv$. 


An {\it involutive automorphism} of $G$ is an element $\theta\in\Aut(G)$
satisfying
$\theta^2(g)=g$ for all $g\in G$.  The group $G$ acts on the space of
involutive automorphisms: $g\cdot\theta=\int(g)\theta\int(g\inv)$.  We
need to understand the structure of the space of orbits in some detail.

Let $\Int(G)$ be the group of inner automorphisms of $G$. The map $g\rightarrow \int(g)$ 
factors to an isomorphism $\Gad\simeq \Int(G)$. 


There is an exact sequence
\begin{equation}
\label{e:autexact}
1\rightarrow \Int(G)\longrightarrow \Aut(G)\overset{p}\longrightarrow \Out(G)\rightarrow 1.
\end{equation}

\begin{definition}
A {\it splitting} of $G$ is a set $\spl=(B,H,\{X_\alpha\})$ where $B$ is
a Borel subgroup, $H\subset B$ is a Cartan subgroup, and $\{X_\alpha\}$ is a set of choices of 
simple root vectors (in the Lie algebra of $G$) for the simple roots of $\Delta^+(H,G)$ (see ?).

Let $\Aut(G,\spl)$ be the automorphisms $\tau$ of $G$ preserving $\spl=(H,B,\{X_\alpha\}$, i.e. satisfying $\tau(H)=H, \tau(B)=B$ and $\tau(\{X_\alpha\})=\{X_\alpha\}$. 

\end{definition}

\begin{theorem}
\begin{enumerate}


\item If $\tau\in \Aut(G,\spl)$ is an inner automorphism then $\tau=1$. 


\item If $\gamma\in\Out(G)$ then there is a unique element $\tau\in\Aut(G,\spl)$ mapping to $\gamma$ in \eqref{e:autexact}.
The map $\gamma\rightarrow s(\gamma)=\tau$ is a splitting of \eqref{e:autexact}.

\item If $\spl'$ is another splitting then there exists a unique inner automorphism $\mu=\tau(g)$ ($g\in G$) so that 
$\spl'=\mu(\spl)$. Hence $\Aut(G,\spl)$ and $\Aut(G,\spl')$ are {\it canonically} isomorphic, via an inner automorphism of $G$.

\item $\Gad\simeq \Int(G)$ acts simply transitively on the set of splittings.

\item $\Out(G)$ is isomorphic to the automorphism group of a based root datum of $G$. 

\item If $G$ is semisimple then $\Out(G)$ is a subgroup of the automorphism group of the Dynkin diagram of $G$,
with equality if $G$ is simply connected or adjoint.
\end{enumerate}
\end{theorem}

\begin{definition}
An {\it inner class} of automorphisms of $G$ is a subset of
$\Aut(G)$ all mapping to the same element of $\Out(G)$.
We say two automorphisms of $G$ are {\it inner to each other} if they are in the same inner class.

If $\gamma\in \Out(G)$ the {\it inner class determined by $\gamma$} is $p\inv(\gamma)\subset \Aut(G)$. 
\end{definition}





